\section{Introduction}


****************** Needs to be updated a lot. The below is old paper's content. Some of it will be useful. I request Sai Koti and Chandramouli to take care of this section.***************************

Electricity is one of the most important components of the modern life. According to a recent survey, there are a total of 18,452 unelectrified villages in India. Providing electricity to these villages is difficult for a number of reasons. The village may be situated very far away from the main grid and it would be difficult to establish a direct electrical line between the main grid and the village. Also, due to increasing global warming, we want to make less use of fossil fuels for the generation of power. Our objective in this work is to provide a solution to electrifying these villages.

The concept of smart grid \cite{weiss1999multiagent}, is aimed at improving traditional power grid operations.  It is a distributed energy network composed of intelligent nodes (or agents) that can either operate autonomously or communicate and share energy. The  power grid is facing wide variety of challenges due to incorporation of renewable and sustainable energy power generation sources. The aim of smart grid is to effectively deliver energy to consumers and maintain grid stability.

A microgrid is a distributed networked group consisting of renewable energy generation sources with the aim of providing energy to small areas. This scenario is being envisaged as an important alternative to the conventional scheme with large power stations transmitting energy over long distances. The microgrid technology is useful particularly in the Indian context where extending power supply from the main grids to remote villages is a challenge. While the main power stations are highly connected, microgrids with local power generation, storage and conversion capabilities, act locally or share power with a few neighboring microgrid nodes \cite{farhangi2010path}.
Integrating microgrids into smartgrid poses several technical challenges. These challenges need to be addressed in order to maintain the reliable and stable operation of electric grid.

Research on smartgrids can be classified into two areas -  Demand-side management and Supply-side management. Demand side management (DSM) (\cite{logenthiran2011multi, wang2010demand,dsm1,dsm2,dsm3,dsm4}) deals with techniques developed to efficiently use the power by bringing the customers into the play. The main idea is to reduce the consumption of power during peak time and shifting it during the other times. This is done by dynamically changing the price of power and sharing this information with the customers. Key techniques to address DSM problem in smart grid are peak clipping, valley filling,  load shifting \cite{maharjan2010demand}. In \cite{reddy2011learned, reddy2011strategy}, Reinforcement Learning (RL) \cite{sutton} is used in smart grids for pricing mechanism so as to improve the profits of broker agents who procure energy from power generation sources and sell it to consumers.  

Supply-side management deals with developing techniques to efficiently make use of renewable and non-renewable energy. In this paper, we consider one such problem of minimizing Demand-Supply deficit in microgrids. 

In our current work, we setup microgrids closer to the villages. These microgrids has power connections from the main grid and also provided with batteries that can store renewable energy sources. Owing to their cost, these batteries will have limited storage capacities. Each microgrid needs to take decision on amount of renewable energy that needs to be used at every time slot and the amount of power that needs to drawn from the main grid. Consider a scenario where, microgrids will use the renewable energy as it is generated. That is, they do not store the energy. Then, during the peak demand, if the amount of renewable energy generated is low and power obtained from the main grid is also low, it leads to huge blackout. Thus, it is important to intelligently store and use the renewable energy. In this work, we apply Multi-agent Q-learning algorithm to solve this problem.

\section*{Organization of the Paper}	
The rest of the paper is organized as follows. In Section 2, we formulate the problem in the framework of MDP. In Section 3, we describe the RL algorithm to solve this problem. In Section 4, we discuss experimental results. In Section 5, we provide concluding remarks followed by future work.